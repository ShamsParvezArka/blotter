\documentclass[12pt, a4paper]{article}
\usepackage{a4wide}
\usepackage{graphicx}
\usepackage{pgfplots}
\usepackage[super]{nth}
\title{Electrical and Electronics Engineering Lab Report}
%\author{Shams Parvez Arka}
\begin{document}
\date{}
\maketitle

\begin{flushleft}
	\textbf{Exp No}: \textit{05}\\
	\textbf{Exp Name}: \textit{\textbf {Verification of Thevenin's Theorem}}.\\ 

	\textbf{Equipmets}: 1.Ammeter 2.Voltmenter 3.Bread borad 4.Three resistance 5.DC power\\ 
	\qquad \qquad \qquad supply 6.Connecting wires
\end{flushleft}

\begin{flushleft}
	\textbf{Theory:} Thevenin's theorem states that a linear two-terminal circuit can be replaced by an equivalent circuit consisting of a voltage source $V_{th}$ in series with a resistor $R_{th}$ where $V_{th}$ is the open-circuit at the terminals and $R_{th}$ is the input or equivalent resistance at the terminals when the independent sources.
\end{flushleft}

\begin{flushleft}
	\textbf{Circuit diagram:} The circuit diagram looks like the following figure below.
	\begin{center}
		\includegraphics[width=70mm]{img/circuit-thevenins-theorem.png}\\
		\textit{Fig 5.1: A circuit diagram}
	\end{center}
\end{flushleft}

\begin{flushleft}
	\textbf{Calculation:} \\
	\qquad \textit{Calculating $R_{th}$}:
\end{flushleft}
\begin{eqnarray*}
	R_{th} &=& (((R_1 || R_2) + R_3) || R_4) R_5 \\
		   &=& (((9.9 || 9.9) + 9.8) || 10) || 9.9 \\
		   &=& 15.859 \Omega \\
\end{eqnarray*}

\begin{flushleft}
	\qquad \textit{Calculating $V_{th}$}:
	At loop 1: 
	\begin{eqnarray*}
		&& -V + I_1R_1 + R(I_1 - I_2) = 0 \\
		&\Rightarrow& I_1(R_1 + R_2) - I_2R_2 = V \\
		&\Rightarrow& 19.8I_1 - 9.9I_2 = 2 \label{eqno} \qquad ... \qquad ... \qquad ... (1)\\
	\end{eqnarray*}
	At loop 2: 
	\begin{eqnarray*}
		&& R_2(I_2 - I_1) + I_2R_3 + R_4I_2 = 0\\
		&\Rightarrow& I_2(R_2 + R_3 + R_4) - I_1R_2 = 0 \\
		&\Rightarrow& 29.7I_2 - 9.9I_1 = 0 \qquad ... \qquad ... \qquad ... (2)\\
	\end{eqnarray*}
	Now solving equation (1) and (2) we get, $ I_1 = 0.121 A,  I_2 = 0.0404 A$ \\
	\qquad \qquad 
	\qquad \qquad 
\end{flushleft}

\begin{flushleft}
	\textbf{Data table:}\\
	\begin{center}
		\begin{tabular}{||l|l|l|l|l|l|l|l|} \hline
			$s.n$ & $V$ & $V_{th}(m)$ & $V_{th}(cal)$ & $R_{th}(m)$ & $R_{th}(cal)$ & $Error_{V_{th}}$ & $Error_{V_{th}}$\\ \hline 
		1 & $2$ & $1.006V$ & $1.278V $ & $16\Omega$ & $15.859\Omega$ & $27.03\%$ & $0.889\%$ \\ \hline
		2 & $3$ & $1.733V$ & $1.9221V$ & $16\Omega$ & $15.859\Omega$ & $10.91\%$ & $0.889\%$ \\ \hline
		\end{tabular}
	\end{center}
\end{flushleft}

\begin{flushleft}
	\textbf{Result:} The experimental results indicates that according to Thevenin's theorem we've solved the above circuit.Though there are some problems in electronics elements of the experiment and these the cause of error. \\
\end{flushleft}

\begin{flushleft}
	\textbf{Discussion:} Thevenin's theorem is very esesntial law for electrical circuit for solving complex circuit diagram. From this experiment we have verified the law if there work accuracy was better. Though there are a little error, we've verified this law though there are a little error. 
\end{flushleft}

\end{document}
